As mentioned in section~\ref{sec:datasets} we will use DPpedia as the main source
for our data about connections.

\paragraph{}
However, since DBpedia is a very general data source it's not optimized
for providing us with movie specific data.
The various namespaces DBpedia uses help us a great deal,
but where the data in DBpedia is missing or incomplete,
we will use the excellent \href{http://trakt.tv/}{trakt.tv} API
to supplement the data we get from DBpedia.

\paragraph{}
Trakt also allows a user to mark any movie they have
already watched as ``seen''.
To save our users the trouble of adding every move they
have watched manually through our interface, we will
develop a trakt activity importer, using the
\texttt{/user/library/movies/watched} endpoint of the
trakt API.
This importer will allow trakt users to select multiple
of their watched movies and import those into the graph.

\paragraph{}
Users of Facebook will be able to import movies they
have marked as liked on their profile.
We will accomplish this using the \texttt{/me/movies}
endpoint of the Graph API.
Similar to the importer for trakt, Facebook users can
select which movies they want to import into the graph
for processing.

\paragraph{}
Using the data from these services in combination with
our primary data source, we aim to create a simple
and fluid user experience that will turn data (a list
of movies) into true information.
