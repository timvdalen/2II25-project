We decided that the focus of our project would be something quite diverse yet described with enough detail on DBpedia to actually be able to properly analyse the data. Since media and in particular movies met those requirements we chose to focus on these. The main idea for our web application is finding links between a set of movies the user likes and find connections and properties shared between the provided films. By using external services that keep track of which movies the user likes, we may be able to start with a good dataset to analyse.

\paragraph{}
We will try to abstract from a particular source of user data by generalizing
an importer dialog and writing a few ``plugins'' for it, that allow the user
to select a data source.
Our initial goal for this project is to create plugins for trakt.tv and
Facebook likes.
We will describe this in more detail in section~\ref{sec:discussion}.

\paragraph{}
The connections between the movies the user likes may be very straightforward and have the same leading actor or be part of the same genre. These common properties may provide a clear view into what the user likes. On the other hand, there may be a film that shares properties with movies the user liked, but the users might not have actually seen and rated it themselves. There might even be some groups of films with a common property that in turn have connections to films which share some properties from both groups. The more common properties the more similar we assume something is but the user might enjoy a hybrid of things he or she likes instead of something that mimics it in every way.

\paragraph{}
Based on the information we can gather from all this we can obviously provide suggestions to the user as to what he or she might like. The provided analysis might also just grant the users some insight in their own choices of movies. A nice graph can be produced showing the found connections between the movies. The user might also request more information on a certain mentioned film besides the relevant properties in connecting the nodes of the produced graph, these can also simply be requested from DBpedia, even though we might not be using that particular property to compare films. We probably will not be comparing the cover art for example, to find a connection between movies, but the user might still be interested in seeing it.

\paragraph{}
In order to maintain a high level of scalability, we will store any generated
data we might need at a later time in the browser of the user using their local
storage.
This way, a user (provided he or she either visits using the same browser or
syncs his or her local storage to another browser) is able to pick up where they
left off, without the need to add all movies to the graph again.
By doing this, we save ourselves from some double processing of movies.
